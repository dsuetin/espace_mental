%%%%%%%%%%%%%%%%%%%%%%%%%%%%%%%%%%%%%%%%%
% Journal Article
% LaTeX Template
% Version 1.3 (9/9/13)
%
% This template has been downloaded from:
% http://www.LaTeXTemplates.com
%
% Original author:
% Frits Wenneker (http://www.howtotex.com)
%
% License:
% CC BY-NC-SA 3.0 (http://creativecommons.org/licenses/by-nc-sa/3.0/)
%
%%%%%%%%%%%%%%%%%%%%%%%%%%%%%%%%%%%%%%%%%

%----------------------------------------------------------------------------------------
%	PACKAGES AND OTHER DOCUMENT CONFIGURATIONS
%----------------------------------------------------------------------------------------

\documentclass[twoside]{article}

\usepackage{lipsum} % Package to generate dummy text throughout this template

\usepackage[sc]{mathpazo} % Use the Palatino font
\usepackage[T1]{fontenc} % Use 8-bit encoding that has 256 glyphs
\linespread{1.05} % Line spacing - Palatino needs more space between lines
\usepackage{microtype} % Slightly tweak font spacing for aesthetics

\usepackage[hmarginratio=1:1,top=32mm,columnsep=20pt]{geometry} % Document margins
\usepackage{multicol} % Used for the two-column layout of the document
\usepackage[hang, small,labelfont=bf,up,textfont=it,up]{caption} % Custom captions under/above floats in tables or figures
\usepackage{booktabs} % Horizontal rules in tables
\usepackage{float} % Required for tables and figures in the multi-column environment - they need to be placed in specific locations with the [H] (e.g. \begin{table}[H])
\usepackage{hyperref} % For hyperlinks in the PDF

\usepackage{lettrine} % The lettrine is the first enlarged letter at the beginning of the text
\usepackage{paralist} % Used for the compactitem environment which makes bullet points with less space between them

\usepackage{abstract} % Allows abstract customization
\renewcommand{\abstractnamefont}{\normalfont\bfseries} % Set the "Abstract" text to bold
\renewcommand{\abstracttextfont}{\normalfont\small\itshape} % Set the abstract itself to small italic text

\usepackage{titlesec} % Allows customization of titles
\renewcommand\thesection{\Roman{section}} % Roman numerals for the sections
\renewcommand\thesubsection{\Roman{subsection}} % Roman numerals for subsections
\titleformat{\section}[block]{\large\scshape\centering}{\thesection.}{1em}{} % Change the look of the section titles
\titleformat{\subsection}[block]{\large}{\thesubsection.}{1em}{} % Change the look of the section titles

\usepackage{fancyhdr} % Headers and footers
\usepackage[utf8x]{inputenc}

\pagestyle{fancy} % All pages have headers and footers
\fancyhead{} % Blank out the default header
\fancyfoot{} % Blank out the default footer
\fancyhead[C]{Influence of Christmas Environment on Customer's Choices $\bullet$ December 2015 $\bullet$} % Custom header text
\fancyfoot[RO,LE]{\thepage} % Custom footer text

\usepackage[russian,english]{babel}
%----------------------------------------------------------------------------------------
%	TITLE SECTION
%----------------------------------------------------------------------------------------

\title{\vspace{-15mm}\fontsize{24pt}{10pt}\selectfont\textbf{Les Fondements Mathématiques de la Psychologie}} % Article title 

\author{Psychology
\large
\textsc{}\thanks{.}\\[2mm] % Your name
\normalsize Daniil Suetin \\
\normalsize \href{mailto:}{} % Your email address
\vspace{-5mm}
}
\date{May 25, 2023}

%----------------------------------------------------------------------------------------

\begin{document}
\selectlanguage{russian}
\maketitle % Insert title
\thispagestyle{fancy} % All pages have headers and footers

%----------------------------------------------------------------------------------------
%	ABSTRACT
%----------------------------------------------------------------------------------------

\begin{abstract}
Процесс мышления издавна занимает мышление человека )))



In this experiment, AP Psychology 5th period asked the question "Does having a Christmas environment make shoppers more likely to choose holiday items rather than year-round ones?" The objective of this experiment was to answer this question. Also, a purpose of this experiment was to see, if the Christmas environment did, indeed, influence purchases, then to what extent? Two JROTC groups from 5th period were randomly selected, made their way to the AP Psychology classroom, and went on a shopping trip. These groups were in two different environments, non-Christmas and Christmas. Each environment had both non-Christmas and Christmas items to choose from. Data was collected from observers, "food cards", as well as pre and post surveys. Their data was collected and analyzed to see if the Christmas environment had any impact on whether students chose the Christmas items. After deep analysis, it can be seen that the Christmas environment influenced student's "purchases" (they did not actually have to buy the items). Although the first group (neutral environment) did not solely choose year-round items and the second group (Christmas environment) did not solely choose Christmas items, differences in purchases between the groups can still be noted. During the discussion with the subjects, some feel as though they were directly influenced by the Christmas decorations or lack thereof, while others say that they chose subconsciously. 

\end{abstract}

%----------------------------------------------------------------------------------------
%	ARTICLE CONTENTS
%----------------------------------------------------------------------------------------

\begin{multicols}{2} % Two-column layout throughout the main article text

\section{Introduction}

\lettrine[nindent=0em,lines=3]{T}he function of this experiment was to determine whether a Christmas environment influences a customer's purchases. This experiment was done on a very small scale, but is still relevant to the question. The question asked in this experiment was an important one because it can be applied on a broader spectrum and can help explain how grocery stores influence customers to buy products that the store wants them to buy. From AP Psychology class, it has been learned already that grocery stores "fool" customers into buying certain things based on the environment, placement, and overall feel of the store. Also, from previous knowledge in class, AP Psychology knows that people tend to grab items with their right hand, so the experiment was set up accordingly. This study will widen the knowledge of how stores influence customers and will help explain to what extent grocery stores influence customers without them knowing. The hypothesis of this experiment was that having a Christmas environment will tend to make students more likely to choose the Christmas items available. The expectation of the experiment was that, although students probably will not choose all non-Christmas items or all Christmas items, there will still be a noticeable difference. The outcomes of this experiment can be (1) a Christmas environment influences customers' choices or (2) a Christmas environment does not influence a customers' choices.


%------------------------------------------------

\section{Methods}
\subsection{Research Methods}
In this experiment, 5th period JROTC was studied. Subjects' ages ranged from 14-17 years old. Students were selected based on their class period correlation with 5th period AP Psychology, which served as a random sampling method.  The study was carried out in Ms. Ahonen's classroom at Ola High School in McDonough, Georgia. This sampling method was a stratified sample. Groups were chosen by randomly handing out colored cards with the words "Group One" and "Group Two" on them. Starting off, there were 24 students. However, only 21 students participated in the experiment due to absences and confusion. Two groups were randomly chosen. Group one included ten students and served as the control group because they shopped in a neutral environment. Group two included eleven students and served as the experimental group because they shopped in a festive environment, with holiday scents, decoration, and music. 
\subsection{Scientific Method}
The independent variable was the environment that the subjects were in. The dependent variable was the types of "purchases" that the subjects made. Data collection occurred in many different ways as to ensure that a sufficient amount of data was gathered for proper study. Subjects were given a pre-survey with questions asking if they were allergic to chocolate, nuts, gluten, fruit, or any other allergies that AP Psychology 5th period should become aware of. This helped determine products that would be appropriate for the grocery store simulation. Each subject also received a "food card" that they could use to get items from the grocery store. When the student got a food item, the item chosen was highlighted. At the end of the experiment, all food cards were gathered for analysis. After the experiment, there was a post-survey. Subjects were asked how they felt during the process and why they chose the items that they did. During the debrief, subjects were asked if they had figured out "the point" of the study. Throughout the entire experiment on December 11, 2015, observers were planted around the room and took notes on occurrences that were seen or heard. This served as an important piece of data collection that gave insight below the surface.
\subsection{Design}
The experiment was designed in such a way as to maximize efficiency and results. First, the classroom was set up in a neutral way. There were four tables set up in a semi circle each with two products (four choices) on each table with the exception of the first table (with only one product). There were no decorations and there was a video of a beach playing in the front. There was also "shopping music" playing in the background.Observers got into their places as well as the students manning the four tables. Two AP Psychology students went and explained the experiment, separated the JROCT group into two groups, gave the first group their "food cards", and brought the first group in. Two students entered at a time as to reduce congestion in the room. Each subject got a plastic bag to put their products in. The subjects went through the grocery store simulation and got a total of five products and had seven minutes to do so. After the group was done shopping, the two AP Psychology students that brought the subjects in, guided them back to their JROCT classroom. During the "flip" AP Psychology turned the classroom into a Christmas environment, with posters, a fireplace video, a Christmas scent, assorted Christmas trays and table cloths,
dimmed lights, Christmas lights,
snowflakes, and a mini Christmas tree. At the same time, Group Two was being given their food cards and brought in. The two groups avoided seeing each other as to avoid talking among the subjects. Group Two entered and did the same as Group One. Two students were brought in at once. Each student got a plastic bag and was given the same amount of time (seven minutes total) to shop for their products. After the seven minutes passed, the two that brought the subjects in, took them back. A few students went down to the JROCT classroom in order to give the subjects a post survey and provide a debrief. The post-survey was filled out and collected to use as data. There was then a discussion among AP Psychology students and the subjects about the point of the study and what AP Psychology was measuring. Observers, again, were planted in the room and were making note of observations throughout the survey and debrief. By completing this experiment, AP Psychology tested the hypothesis and collected data to support it. In this study, both quantitative and qualitative data was collected. Quantitative data includes the food cards, while qualitative data includes the surveys and observations. The data collected was deemed "significant" if it gave AP Psychology information that we had not known before. It did not matter if our hypothesis was right or not, but most all data collected was significant. To analyze the data, charts and graphs were created. This made an analysis easier.
\subsection{Prior Research}
There have been a multitude of studies based on how Christmas music makes people feel and whether or not it affects the things that we buy. Scholars have investigated the effects that Christmas music has on retail store customers (Nierenberg 2012). Christmas music effects a customer's cognitive health, making the customer lose sense of their time and, in result, cause them to shop longer and buy more things. Coinciding with these studies there have also been studies regarding how Christmas scents make people feel (Spangenberg, Grohman, Sprott 2005). Research suggests that when stores expose their customers to musical stimuli and scents, they both influence how a customer feels while shopping. It was also discovered that when a customer was exposed to both Christmas music and a Christmas scent then the customer acts favorably towards Christmas merchandise. Christmas music is not the only thing that effects people. The pace of the music in general is a factor tat determines how fast a customer shops (Spangenberg 2005). Slower paced music causes customers to slow down and take their time while shopping, while fast paced music results in customers moving faster throughout the store. Lighting is also something that has a big effect on customers in a store. Since the lighting changed in the experiment, it certainly could have been an important factor attributing to our results. Lighting can affect emotions, attention towards merchandise, and behaviors in the store (Lam 2001). Also, customers tend to grab with their right hands. Because of this, grocery stores set up the shelves so customers grab more, expensive items. Our eyes always lean towards the right side (Johnson 2013)
so it was important to put the generic items on the right to see if the subjects really went against nature and grabbed the Christmas item.


%------------------------------------------------

\section{Results}
After the experiment was completed, the food cards were compiled and results became evident. In Group One, the control group, three people chose the Christmas Coke while six people chose the regular coke. One person chose a Christmas cookie while six chose a regular cookie. Four chose a candy cane and four chose mints. Four chose a Christmas brownie and six chose a regular brownie. Two chose a Christmas water and three chose a regular water. One chose Christmas M and M's while four chose regular M and M's. Four chose Christmas Hershey Kisses and four chose regular Hershey Kisses. However, a distinct difference can be noted when Group Two chose their items. Seven people chose the Christmas Coke and only one person chose the regular Coke. Five people chose the Christmas cookies and five people chose the regular cookies. Eight people chose a candy cane and no one chose mints. Five people chose Christmas brownies and six people chose regular brownies. One person chose the Christmas water and two people chose the regular water. Three people chose Christmas M and M's and four people chose regular M and M's. Three people chose Christmas kisses and five people chose regular kisses.\\ \\ \\ \\ \\ \\ \\ \\ \\ \\ \\
\\ \\ \\ \\ \\ \\ \\ \\ \\ \\ \\ \\ \\ \\ \\ 
\\ \\ \\ \\ \\ \\ \\ \\ \\ \\ \\ \\ \\ \\ \\ \\ \\
\\ \\ \\ \\ \\ \\ \\ \\ \\ 




%------------------------------------------------



\section{Discussion}

\subsection{Biological Processing}
This experiment deals with sensory transduction because of the testing of the participants senses of the environment and how if affected their decisions. Sensory transduction is the transforming of stimulus energies into neural impulses our brain can interpret. This takes place in the thalamus of our brain. The thalamus is the brain's sensory switchboard, which is located on top of the brain stem. Transduction includes all senses, but the thalamus does not include smell. Senses that were tested in this experiment involving transduction in the thalamus included sight, and auditory. The experimental group had a different environment with Christmas decorations and music which was transduced in their thalamus and sent out as neural impulses. The thoughts that formed from the transduction possibly affected what decisions the participants made. This experiment also dealt with memory associations made in the brain. Memory is an indication that learning has persisted overtime. Visual and emotional memory usually happens in the hypothalamus. Because of visual and acoustic encoding that has happened, the participants may have experienced state dependent or mood congruent memory. State dependent memory is memory that is recalled under the consciousness conditions it was formed. If one of the participants had a past experience with Christmas, the experimental group could have recalled past memories and those memories could have affected if the participants chose Christmas or non-Christmas items. The participants could also have been affected by mood congruent memory. Mood congruent memory is the tendency to recall experiences that are consistent with your current mood. If the participants were in a bad mood, they could have remembered a bad Christmas experience, and chosen not to get Christmas related items. 


\subsection{Sensation/Perception}

In terms of sensation and perception, sensation is comparable to the five sides of a pentagon with perception being the pentagon and the senses being the whole. You cannot keep the shape or have the whole without the sides being together. The goal of the experiment at hand was to see how the environment affected whether people chose Christmas items or non-Christmas items. The first sense that was affected with the study was smell. In the control group, no scents were added to the room so in all actuality, they found themselves in a neutral environment on that front. However, for the second group involved which was the independent variable, the smell of the room was manipulated with the addition of a cinnamon scent. The next sense in line was sight. For the control group, no decorations were visible, and along with that, the lights were on making it to be a more sterile and normal school environment. However, for the manipulated group it was similar with what happened before, Christmas decorations were put up and the lights were dimmed to form a more "Christmas" atmosphere. Moving forward, the same thing was done with hearing as well. For the control group, secular music was played making the room to feel as nonpartisan as possible. For the manipulated group, Christmas music was played in order to further add to the atmosphere and influence their choice. With the sense of feeling and taste, it becomes more complicated, starting with feeling. Nothing could have been done within the allotted time to "flip" the room from secular to Christmas, to affect how the room and items the participants interacted with felt, thus negating this sense as a source of opinion in terms of perception. With the final sense being taste, this one is also different from the first three mentioned. The taste of the foods offered to the participants was kept the same among each choice, only the aesthetics of the items were changed. This was to ensure that the participants chose each item based off the environment of the room, not based upon their own personal preference. Perception of an environment is based upon the information gained from the environment. With this experiment, the subject being studied was the environment and how that affected the overall perception of the room and how it affected their choices.

\subsection{Levels of Consciousness}

Consciousness is defined as one's overall awareness of their environment. In the experiment, the "shoppers" were either aware or unaware of different elements of their environment, be it Christmas music, decorations, or lack thereof. Whether 
or
not people are aware of different stimuli in their environment, these stimuli may still affect the person. Such was the case in this experiment, as while some noticed the placement of Christmas decorations and music, others 
did
not indicate on the exit survey that they noticed these details. Despite this unawareness, the results still reflected that with some products, those in the Christmas environment chose Christmas over secular. These results exemplify the fact that stimuli has affects on one's brain despite their state of consciousness regarding the stimuli.

\subsection{Debrief, Observational Notes, and Post-Survey}
After the experiment concluded, the subjects were given a post-survey to be completed and collected for data. The overall ideas and responses on the surveys range from very insightful to no effort put in. Group One tended to notice that there was a beach on the screen, that there were no desks and that the room seemed generally empty. The reason that Group One chose what they did was because it looked "pretty", because "it would have been cheaper than the Christmas stuff", and because the subject "was hungry". Group One generally felt like the room was a bit intimidating. Specific subjects felt "weird", "pressured", "like they were being watched", and "confused". Group One thought that the point of the experiment was to "see who would grab the Christmas stuff" and to "see if the Christmas items were more appealing". Group Two offered several different responses to the post-survey. Group Two noticed that the room was darker, there were snowflakes on the ground, and that the room was decorated like Christmas. Group Two chose items because they "looked better", because a subject "loves Christmas", because it was "colorful" and "because the atmosphere was seasonal". Those in Group Two felt like it was "a gift shop", and generally felt "happy", "excited", "welcomed" and a bit "mysterious". Group Two said that they thought that the point of the experiment was to "see how the environment changes your mood", 
and to see
"whether or not holiday themes affected choices"

There are many observations that were noted that can be deemed significant in this study that may or may not have affected the subjects' choices. Someone thought that the wrapped Coke was a Diet Coke. Generally, people shopped a bit slower during the Christmas theme, perhaps to stop and look around or stay and smell the Christmas scent. One subject in Group two said "I'm choosing all Christmas!" Also, a subject in Group two asked a friend who got a wrapped Coke "Why did you get that one?" Another important note is that the subjects tended to use their right hand most often. One in Group two took a wrapped Coke then quickly switched to a Christmas Coke. 

During the debrief, AP Psychology students told the subjects the "point" of the experiment. Some seemed shocked, while others were not as surprised. One person said that it was "nice" of AP Psychology to let them shop for the items. During the debrief, the subjects were very talkative and wanted to know what their friends had chosen. 



\subsection{Anomalies}
Some anomalies occurred that are worth noting for the good of this experiment. In Group Two, a subject noticed the point of the experiment and noted that he would not choose Christmas items because he knew that we "wanted him to". Also, the Cokes had to be covered. When bringing in items, AP Psychology thought that one package of Cokes was Christmas and another was plain, but both were Christmas. AP Psychology then had to cover one set with paper to provide a "plain Coke" for choosing. Also, there were a few things that happened in the actual experiment on AP Psychology's part that could have influenced the experiment in some way. Group Two came into the room three minutes before scheduled. Although the "flippers" were finished flipping, this occurrence caused the time table to be a little off. Group Two also went back to their classroom a little late. An anomaly that occurred out of AP Psychology's control was that two students were absent. One student also forgot to leave with Group One. The study started with 24 and ended with 21. 

\section{Conclusion}

\subsection{Overall Worth}
The overall worth of this experiment was gaining new information on consumer shopping habits. Although this was done on a small scale, it was useful as an AP Psychology class. 
AP Psychology
learned the process that goes into conducting an experiment, which gives a broader understanding to how exactly psychologists do their work. Having to conduct an
experiment helped put into action things that were
learned, including variables and ethic principles. Our proposed hypothesis proved correct, which gives  insight on how holiday themed decorations can affect what consumers, even ourselves, choose
to buy. The information can give worth to further studies done in this category.
\subsection{Proving the Hypothesis}

The results of 
the
experiment showed that the experimental group that had the Christmas environment more frequently picked the items related to Christmas. This seems to prove our hypothesis that a Christmas environment has influence over the items the participants will chose in a shopping experience. Although our results were what were
wanted, to be able to say this on a more certain note, more experiments would have to be conducted to prove that this is stable trend and is not just related to preference. For the
small scale experiment, what was
originally predicted came true. The 5th period AP Psychology class makes a prediction that if this experiment was replicated, this hypothesis would prove correct once again, further supporting our findings of this experiment. 
\subsection{Future Studies}

If this experiment was to be repeated with new participants there would be some things would 
be
altered. First, 
it
would be 
made sure there was a more set time table so that no one would come in early. Another thing would change would be to test different age groups, seeing if it had the same effect among people of different ages. Because of participant discussion, 
it was
learned Christmas decorated bags were harder to see in, so an alternative bag that made the items more visible to the participants would be used.





%----------------------------------------------------------------------------------------
%	REFERENCE LIST
%-----------------------------------------



%----------------------------------------------------------------------------------------

\end{multicols}

\end{document}
